
\frontmatter{}

\selectlanguage{english}%
\cleardoublepage{}
\begin{abstract}
相对论重离子碰撞实验的主要目的是研究有限温度和密度下QCD物质的性质。近年来,RHIC和LHC的实验通过对末态强子的测量结果表明,在高能重离子碰撞中形成了一种新的物质形态——强耦合的夸克胶子等离子体(sQGP)。双轻子作为一种电磁探针,由于其不参与强相互作用,因此在产生之后,它们能不受影响的穿过高能重离子碰撞形成的介质。同时,双轻子能够在系统演化的各个阶段产生,使它们成为研究高能重离子碰撞中产生的介质性质的优秀探针。

由于关注的物理不同,双轻子的动力学相空间一般被分为3个不变质量质量区间。低质量区间($M_{ll}<M_{\phi}$),双轻子的产生主要来源于介子在强子介质中的多次散射。在中间质量区间($M_{\phi}<M_{ll}<M_{J/\psi}$),双轻子的主要来源于QGP物质的热辐射,以及重味夸克介子的半轻子衰变。在高质量区间($M_{ll}>M_{J/\psi}$),双轻子主要来源于Drell-Yan过程和重味夸克偶素的衰变。

本文主要讨论STAR实验组在$\sqrt{s}$ = 200 GeV下得质子质子对撞和金金对撞中对双电子产生的测量。STAR实验在2010至2012年间积累了大量的实验数据,同时桶部飞行时间探测器(TOF)的安装完成大大增强了STAR探测器鉴别电子(正电子)的能力,使得在STAR实验中测量双轻子成为可能。通过分析2012年采集数据的到的双电子质量普极大地提高了质子质子对撞中双轻子不变质量谱的精度(与STAR实验之前的测量\cite{PhysRevC.86.024906}比较,统计量提高了7倍左右)。通过在200GeV金金碰撞中双轻子不变质量谱与强子衰变模拟和模型计算的比较,我们发现,在质量区间0.3\textasciitilde{}0.76
GeV/$c^{2}$ 中,双轻子的产生相对于不包含$\rho$介子贡献的强子衰变模拟结果有 $1.66\text{\ensuremath{\pm}0.06(stat.)\ensuremath{\pm}0.24(sys.)\ensuremath{\pm}0.41(cocktail)}$
倍的增强。这个增强因子比PHENIX实验观测到的结果要小。基于$\rho$介子在介质中质量谱展宽的模型计算可以很好的解释我们观测到的增强。同时本文也会讨论可能存在的相关量重味夸克在介质中的修正效应,以及通过双轻子衰变道对$\omega$介子和$\phi$介子产生的测量。

\keywords{双轻子,矢量介子,手征对称性恢复}
\end{abstract}
\cleardoublepage{}
\begin{eabstract}
The mission of an ultra relativistic heavy ion program is to study
QCD matter at finite temperature and density in the laboratory. Many
experimental evidences have been found to demonstrate the formation
of a strongly-coupled Quark Gluon Plasma (sQGP). These evidences are
mostly from hadron measurements in high energy heavy ion collision
at RHIC and LHC. Dileptons as an electromagnetic probe, escape the
interacting system without suffering further strong interactions after
their production. In additional, dilepton can be produced on the various
stages of entire system evolution. They are therefore expected to
be an outstanding probes to study the property of the medium created
in high energy heavy ion collisions. 

Traditionally, due to different physics of interest, dielectron kinematic
phase space is divided into 3 mass regions. In the Low Mass Region
- LMR ($M_{ll}<M_{\phi}$), dileptons are produced via multiple hadron-hadron
scattering by coupling to vector mesons. In the Intermediate Mass
Region - IMR ($M_{\phi}<M_{ll}<M_{J/\psi}$), dilepton production
are dominated by contributions from the thermal radiation and contributions
from semilepton decay of the correlated charmed meson. Drell-Yan process
and the decay heavy quarkonia contribute mainly for the dilepton source
in the High mass Region - HMR ($M_{ll}>M_{J/\psi}$).

In this thesis, I will report the measurement on dielectron production
in p+p and Au+Au collisions at $\sqrt{s}$ = 200 GeV for the STAR
experiment. The large data set collected during year 2010\textasciitilde{}2012,
as well as the complication of barrel Time-Of-Flight (TOF) installation
in year 2010, provides electron/position samples with high statistics
and purity which makes it possible for dielectron measurement at the
STAR experiment. The dielectron results from 200 GeV p+p collision
has greatly improved (\textasciitilde{}7 times more) statistics comparing
to the previous published result from STAR \cite{PhysRevC.86.024906}.
The results from Au+Au collision at $\sqrt{s_{NN}}$ = 200 GeV are
compared with hadronic decay cocktail and model calculations. We observed
an enhancement factor of \foreignlanguage{american}{$1.66\pm0.06(stat.)\pm0.24(sys.)\pm0.41(cocktail)$}
in mass region 0.3\textasciitilde{}0.76 GeV/$c^{2}$ when comparing
to the hadronic cocktail without $\rho$ contribution. The enhancement
is smaller than the PHENIX results. The model calculation based on
in-medium broadening of $\rho$ spectra function can reproduce the
enhancement. I will also report the study of possible modification
of the correlated charm and the measurements of $\omega$ and $\phi$
production though their dielectron decay channel. 

\ekeywords{Dilepton,vector meson, chiral symmtry }
\end{eabstract}
\cleardoublepage{}

\tableofcontents{}

\listoftables


\addcontentsline{toc}{chapter}{List of Tables}%加入目录

\listoffigures


\addcontentsline{toc}{chapter}{List of Figures}%加入目录





\cleardoublepage{}


\begin{description}
\item \end{description}

